\documentclass[11pt,a4paper,sans]{moderncv} % Font sizes: 10, 11, or 12; paper sizes: a4paper, letterpaper, a5paper, legalpaper, executivepaper or landscape; font families: sans or roman
\usepackage{standalone}
\moderncvstyle{classic} % CV theme - options include: 'casual' (default), 'classic', 'oldstyle' and 'banking'
\moderncvcolor{blue} % CV color - options include: 'blue' (default), 'orange', 'green', 'red', 'purple', 'grey' and 'black'

\usepackage{lipsum} % Used for inserting dummy 'Lorem ipsum' text into the template

\usepackage[scale=0.85]{geometry} % Reduce document margins
%\setlength{\hintscolumnwidth}{3cm} % Uncomment to change the width of the dates column
%\setlength{\makecvtitlenamewidth}{10cm} % For the 'classic' style, uncomment to adjust the width of the space allocated to your name

%\usepackage[utf8]{inputenc}

%\usepackage{booktabs}
\usepackage{fontawesome}
\usepackage{marvosym} % For cool symbols.
%\usepackage{hyperref}




\firstname{Jose} % Your first name
\familyname{Perdiguero} % Your last name

% All information in this block is optional, comment out any lines you don't need
\title{Curriculum Vitae}
\address{Universitat de València}{Physics department}
%\mobile{(+91) 0000000, (+91) 0000000}


%\fax{(000) 111 1113}
 
%\social{github}{stefano-bragaglia}
\email{jose.perdiguero@gmail.com} 



%\homepage{www.xyz.com/}{My Webpage}

% social link \faGithub, \faSkype, \faLinkedin,\faStackExchange, and \faStackOverflow
%\extrainfo{
 %   \faGithub\href{https://github.com/xyz}{ Github} \quad
  %  \faLinkedin\href{https://www.linkedin.com/abc/}{ Linkedin} \quad
   % \faSkype\href{https://skype.com/abc}{Skype}
    %}



%\social[linkedin][www.linkedin.com]{name}
% The first argument is %the url for the clickable link, the second argument is the url displayed in the %template - this allows special characters to be displayed such as the tilde in this %example

%\photo[70pt][0.3pt]{picture} % The first bracket is the picture height, the second is %the thickness of the frame around the picture (0pt for no frame)
%\quote{Not Attention, Patience is all we need.}

%----------------------------------------------------------------------------------------

\newcommand{\cvdoublecolumn}[2]{%
  \cvitem[.75em]{}{%
    \begin{minipage}[t]{\listdoubleitemcolumnwidth}#1\end{minipage}%
    \hfill%
    \begin{minipage}[t]{\listdoubleitemcolumnwidth}#2\end{minipage}%
    }%
}



\usepackage{multibbl}
\newcommand\Colorhref[3][orange]{\href{#2}{\small\color{#1}#3}}


\begin{document}

\makecvtitle % Print the CV title

\section{Research interests}

I am a theoretical physicist interested in the formulation and study of modify/alternative gravitational theories with applications to cosmology such as: inflation and dark energy. I am also interested in the study of black holes and gravitational waves.\\

I am currently working on three different projects in the context of Polynomial Affine Gravity, which is an alternative gravitational model, where the gravitational interactions are solely mediated by the affine connection, and the metric tensor is completely exclude in this gravitational formulation.\\

Because of the complexity of the field equations, I am currently studying numerical methods to solve differential equations and partial differential equations using Python libraries. In particular, I am interested in the use of A.I. such as neural network to solve numerically the field equations. 

\section{Appointments}
\cvitem{2018 -- 2023}{\textit{Scientific Research}, Federico Santa Maria Technical University, Chile.}
\cvitem{2015 -- 2019}{\textit{Teaching Assistant}, Federico Santa Maria Technical University, Chile.}


\section{Education}

\cventry{2023--2026}{PhD Physics student}{Universitat de València}{Spain}{}{} 

\cventry{2021--2022}{PhD Physics student}{Federico Santa Maria Technical University}{Chile}{}{I completed five semesters before leaving the program.} 

\cventry{2017--2019 :}{Master of Science, Physics}{Federico Santa Maria Technical University}{Chile}{}{}

\cventry{2014--2016 :}{Bachelor of Science, Physics}{Federico Santa Maria Technical University}{Chile}{}{}


\section{Scholarships \& Awards}

\cvitem{2018 -- 2019}{\textit{\textbf{PIIC}} (Programa de incentivos a la investigacion Cientifica) scholarship, Federico Santa Maria Technical University.}
\cvitem{2017 -- 2019}{\textit{\textbf{Full M.Sc. tuition fees scholarship}}, Federico Santa Mar\'ia Technical University}
\cvitem{2017 -- 2019}{\textit{\textbf{Full M.Sc. stipend scholarship}}, Federico Santa Mar\'ia Technical University}




\section{Publications}
\subsection{Preprints}
\newbibliography{preprints}
\nocite{preprints}{*}
\bibliographystyle{preprints}{plainyrrev}
\bibliography{preprints}{preprints}
{\large \textsc{Refereed Preprints Publications}}


\subsection{Journa Articles}
\newbibliography{journal}
\bibliographystyle{journal}{plainyrrev}
\nocite{journal}{*}
\bibliography{journal}{journal}
{\large \textsc{Refereed Journal Articles}}

\subsection{Books}
\newbibliography{Books}
\nocite{Books}{*}
\bibliographystyle{Books}{plainyrrev}
\bibliography{Books}{Books}
{\large \textsc{Refereed Books Publications}}

\section{Current research projects}


\cventry{10-2023 - 04-2024}{\textit{Space-time classification of exact cosmological solutions to three-dimensional polynomial affine model of gravity.}}{The project study all analytic solutions to the field equations in the context of cosmology, and defined the type of space according to the structure of emergent metric tensor and non-metricity.}{}{}{}

\cventry{10-2023 - 2-2024}{\textit{Review of cosmological solutions in the four-dimensional polynomial affine model of gravity.}}{By using the results obtained from the model in $2+1$ dimensions, we are able to reduce the complexity of the field equations and introduce the effects of the torsion. We study study analytic solutions with all fundamental fields turned on}{}{}{}

\cventry{01-2024 - 04-2024}{\textit{Spherical solutions to polynomial affine model of gravity in the torsion free sector}}{In this project we explore the space of static spherical solutions to the polynomial affine model of gravity in the torsion-free limit. Because of the structure of the field equations we explore numerical methods to find solution using neural networks, specifically PINNs (Physics Informed Neural Network)}{}{}{}




\section{Monograph}

\cventry{09-2019}{\textit{Cosmologia en gravedad afin polinomial (in Spanish)}}{M.Sc. in Physics Thesis, UTFSM }{Chile}{https://repositorio.usm.cl/handle/11673/48007/}{}


\section{Conferences}

\cventry{12-2023}{\textit{Cosmological solutions with torsion effects in Polynomial Affine Gravity}}{7th Workshop Universidad de Valencia}{Spain}{}{}
\cventry{05-2023}{\textit{An inflationary scenario in an effective polynomial affine model of gravity}}{Panoramas UTFSM-PUCV}{Chile}{}{}
\cventry{11-2022}{\textit{Polynomial Affine Gravity in 3+1 dimensions}}{Sochifi (Chilean Physics' Society)}{Chile}{}{}
\cventry{10-2022}{\textit{Cosmology in Polynomial Affine Gravity in 3 + 1 dimensions with torsion}}{Panoramas UTFSM-PUCV}{Chile}{}{}
\cventry{11-2021}{\textit{Cosmology in Polynomial Affine Gravity in 2 + 1 dimensions with torsion}}{Panoramas UTFSM-PUCV}{Chile-via zoom}{}{}
\cventry{09-2021}{\textit{Polynomial Affine Gravity}}{Alternative Gravities and Fundamental Cosmology}{Poland-via zoom}{}{}



\section{Schools}
\cventry{09 - 2023}{3rd Winter School: Topics on Graviticulas}{Pontificie Universidad Cat\'olica de Chile}{Chile}{}{}
\cventry{03-2020}{\textit{School of Classical and Quantum Black Holes}}{University of Concepcion}{Chile}{}{}
\cventry{03-2020}{\textit{School of Gravity and General Relativity}}{CECs (Centro de Estudios Cientificos)}{Chile}{}{}


\section{Teaching Assistant}

\cventry{2017 -- 2019:}{Physics IV}{}{Department of Physics, Federico Santa Maria Technical University}{}{}
\cventry{2015 -- 2016:}{Intermediate Mechanics I}{}{Department of Physics, Federico Santa Maria Technical University}{}{}
\cventry{2016}{Optics}{}{Department of Physics, Federico Santa Maria Technical University}{}{}

\section{Computer skills}

\cvitem{Languages}{SageMath, Cadabra, Mathematica, LaTeX and Overleaf, Currently learning Python and its libreries numpy, matplotlib applied to solve numerically systems of differential equations and dynamical systems}

\cvitem{CodeCademy}{Learn Python 3, Learn Statistics with Python, Learn Statistics with Numpy}

\section{Languages}
\cventry{Spanish}{Native}{}{}{}{}
\cventry{English}{C1}{IELTS-Academic Test}{7.5}{}{}


\end{document}